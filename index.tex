% Options for packages loaded elsewhere
\PassOptionsToPackage{unicode}{hyperref}
\PassOptionsToPackage{hyphens}{url}
\PassOptionsToPackage{dvipsnames,svgnames,x11names}{xcolor}
%
\documentclass[
  letterpaper,
  DIV=11,
  numbers=noendperiod]{scrreprt}

\usepackage{amsmath,amssymb}
\usepackage{iftex}
\ifPDFTeX
  \usepackage[T1]{fontenc}
  \usepackage[utf8]{inputenc}
  \usepackage{textcomp} % provide euro and other symbols
\else % if luatex or xetex
  \usepackage{unicode-math}
  \defaultfontfeatures{Scale=MatchLowercase}
  \defaultfontfeatures[\rmfamily]{Ligatures=TeX,Scale=1}
\fi
\usepackage{lmodern}
\ifPDFTeX\else  
    % xetex/luatex font selection
\fi
% Use upquote if available, for straight quotes in verbatim environments
\IfFileExists{upquote.sty}{\usepackage{upquote}}{}
\IfFileExists{microtype.sty}{% use microtype if available
  \usepackage[]{microtype}
  \UseMicrotypeSet[protrusion]{basicmath} % disable protrusion for tt fonts
}{}
\makeatletter
\@ifundefined{KOMAClassName}{% if non-KOMA class
  \IfFileExists{parskip.sty}{%
    \usepackage{parskip}
  }{% else
    \setlength{\parindent}{0pt}
    \setlength{\parskip}{6pt plus 2pt minus 1pt}}
}{% if KOMA class
  \KOMAoptions{parskip=half}}
\makeatother
\usepackage{xcolor}
\setlength{\emergencystretch}{3em} % prevent overfull lines
\setcounter{secnumdepth}{5}
% Make \paragraph and \subparagraph free-standing
\ifx\paragraph\undefined\else
  \let\oldparagraph\paragraph
  \renewcommand{\paragraph}[1]{\oldparagraph{#1}\mbox{}}
\fi
\ifx\subparagraph\undefined\else
  \let\oldsubparagraph\subparagraph
  \renewcommand{\subparagraph}[1]{\oldsubparagraph{#1}\mbox{}}
\fi


\providecommand{\tightlist}{%
  \setlength{\itemsep}{0pt}\setlength{\parskip}{0pt}}\usepackage{longtable,booktabs,array}
\usepackage{calc} % for calculating minipage widths
% Correct order of tables after \paragraph or \subparagraph
\usepackage{etoolbox}
\makeatletter
\patchcmd\longtable{\par}{\if@noskipsec\mbox{}\fi\par}{}{}
\makeatother
% Allow footnotes in longtable head/foot
\IfFileExists{footnotehyper.sty}{\usepackage{footnotehyper}}{\usepackage{footnote}}
\makesavenoteenv{longtable}
\usepackage{graphicx}
\makeatletter
\def\maxwidth{\ifdim\Gin@nat@width>\linewidth\linewidth\else\Gin@nat@width\fi}
\def\maxheight{\ifdim\Gin@nat@height>\textheight\textheight\else\Gin@nat@height\fi}
\makeatother
% Scale images if necessary, so that they will not overflow the page
% margins by default, and it is still possible to overwrite the defaults
% using explicit options in \includegraphics[width, height, ...]{}
\setkeys{Gin}{width=\maxwidth,height=\maxheight,keepaspectratio}
% Set default figure placement to htbp
\makeatletter
\def\fps@figure{htbp}
\makeatother
% definitions for citeproc citations
\NewDocumentCommand\citeproctext{}{}
\NewDocumentCommand\citeproc{mm}{%
  \begingroup\def\citeproctext{#2}\cite{#1}\endgroup}
\makeatletter
 % allow citations to break across lines
 \let\@cite@ofmt\@firstofone
 % avoid brackets around text for \cite:
 \def\@biblabel#1{}
 \def\@cite#1#2{{#1\if@tempswa , #2\fi}}
\makeatother
\newlength{\cslhangindent}
\setlength{\cslhangindent}{1.5em}
\newlength{\csllabelwidth}
\setlength{\csllabelwidth}{3em}
\newenvironment{CSLReferences}[2] % #1 hanging-indent, #2 entry-spacing
 {\begin{list}{}{%
  \setlength{\itemindent}{0pt}
  \setlength{\leftmargin}{0pt}
  \setlength{\parsep}{0pt}
  % turn on hanging indent if param 1 is 1
  \ifodd #1
   \setlength{\leftmargin}{\cslhangindent}
   \setlength{\itemindent}{-1\cslhangindent}
  \fi
  % set entry spacing
  \setlength{\itemsep}{#2\baselineskip}}}
 {\end{list}}
\usepackage{calc}
\newcommand{\CSLBlock}[1]{\hfill\break\parbox[t]{\linewidth}{\strut\ignorespaces#1\strut}}
\newcommand{\CSLLeftMargin}[1]{\parbox[t]{\csllabelwidth}{\strut#1\strut}}
\newcommand{\CSLRightInline}[1]{\parbox[t]{\linewidth - \csllabelwidth}{\strut#1\strut}}
\newcommand{\CSLIndent}[1]{\hspace{\cslhangindent}#1}

\KOMAoption{captions}{tableheading}
\makeatletter
\@ifpackageloaded{bookmark}{}{\usepackage{bookmark}}
\makeatother
\makeatletter
\@ifpackageloaded{caption}{}{\usepackage{caption}}
\AtBeginDocument{%
\ifdefined\contentsname
  \renewcommand*\contentsname{Table of contents}
\else
  \newcommand\contentsname{Table of contents}
\fi
\ifdefined\listfigurename
  \renewcommand*\listfigurename{List of Figures}
\else
  \newcommand\listfigurename{List of Figures}
\fi
\ifdefined\listtablename
  \renewcommand*\listtablename{List of Tables}
\else
  \newcommand\listtablename{List of Tables}
\fi
\ifdefined\figurename
  \renewcommand*\figurename{Figure}
\else
  \newcommand\figurename{Figure}
\fi
\ifdefined\tablename
  \renewcommand*\tablename{Table}
\else
  \newcommand\tablename{Table}
\fi
}
\@ifpackageloaded{float}{}{\usepackage{float}}
\floatstyle{ruled}
\@ifundefined{c@chapter}{\newfloat{codelisting}{h}{lop}}{\newfloat{codelisting}{h}{lop}[chapter]}
\floatname{codelisting}{Listing}
\newcommand*\listoflistings{\listof{codelisting}{List of Listings}}
\makeatother
\makeatletter
\makeatother
\makeatletter
\@ifpackageloaded{caption}{}{\usepackage{caption}}
\@ifpackageloaded{subcaption}{}{\usepackage{subcaption}}
\makeatother
\ifLuaTeX
  \usepackage{selnolig}  % disable illegal ligatures
\fi
\usepackage{bookmark}

\IfFileExists{xurl.sty}{\usepackage{xurl}}{} % add URL line breaks if available
\urlstyle{same} % disable monospaced font for URLs
\hypersetup{
  pdftitle={Ecologie spatiale et régime alimentaire d'une espèce en danger critique d'extinction : apports pour la conservation du Vison d'Europe (Mustela lutreola).},
  pdfauthor={Rémi Bodinier},
  colorlinks=true,
  linkcolor={blue},
  filecolor={Maroon},
  citecolor={Blue},
  urlcolor={Blue},
  pdfcreator={LaTeX via pandoc}}

\title{Ecologie spatiale et régime alimentaire d'une espèce en danger
critique d'extinction : apports pour la conservation du Vison d'Europe
(\emph{Mustela lutreola}).}
\author{Rémi Bodinier}
\date{2024-03-27}

\begin{document}
\maketitle

\renewcommand*\contentsname{Table of contents}
{
\hypersetup{linkcolor=}
\setcounter{tocdepth}{2}
\tableofcontents
}
\bookmarksetup{startatroot}

\chapter*{Preface}\label{preface}
\addcontentsline{toc}{chapter}{Preface}

\markboth{Preface}{Preface}

Ce projet de thèse est réalisé par Rémi Bodinier au sein du GREGE
(Groupe de Recherche et d'Etude pour la Gestion de l'environnement) en
codirection avec le LBBE (Laboratoir de Biométrie et Biologie
Evolutive).

Le but de cette thèse est d'apporter des connaissances sur l'écologie du
Vison d'Europe à travers l'occupation de l'espace, l'utisation de
l'habitat, le régime alimentaire et les risques de collisions routières.
Grâce à cet apport de connaissances, le but de cette thèse est
d'améliorer les stratégies de conservation de l'espèce en milieu naturel
et de donner des critères pour une futur translocations d'individus dans
le milieu naturel.

\bookmarksetup{startatroot}

\chapter{Introduction}\label{introduction}

La récente et importante perte de biodiversité est un phénomène qui
suscite des inquiétudes de plus en plus nombreuses. Aussi alarmant, de
plus en plus d'espèces voient leur statut de conservation devenir chaque
année plus préoccupant (Ceballos et al., 2010) et le Vison d'Europe fait
partie de ces espèces dont le statut est particulièrement inquiétant. En
effet, cette espèce, considérée comme le mammifère carnivore le plus
menacé en Europe, est classée « en danger critique d'extinction » sur
les listes rouges mondiale (2011), européenne (2011), française (2017)
de l'UICN (MNHN, 2023). Les raisons de son déclin sont nombreuses mais
nous pouvons noter en particulier la mortalité accidentelle par
collisions routières qui joue aujourd'hui un rôle majeur en créant des
puits de mortalité locaux sur les derniers noyaux de population présents
(dans le cadre du premier PNA Vison, 69 spécimens découverts morts ont
été collectés dont 62\% (43) étaient victimes de collisions routières ;
Mission Vison d'Europe, 2003). Nous pouvons citer aussi la perte,
dégradation et fragmentation de ses habitats (Maran et Henttonen, 1995 ;
Lodé, 2001 ; DREAL et al., 2021), l'expansion d'une espèce exotique
envahissante, le Vison d'Amérique (Maran et al., 1998 ; Sidorovich, 2001
; DREAL et al., 2021), ainsi que l'action de certains agents pathogènes
virulents (Fournier-Chambrillon et al., 2022). En 2007, on estimait que
ces menaces ont conduit à la perte de 85\% de l'aire de répartition
d'origine de l'espèce et plus de 90\% de ses effectifs d'origine. En
France, l'aire de répartition de l'espèce est passée de 38 départements
à la fin du XIXème siècle à seulement sept au début du XXIème (Figure 1)
et le nombre d'individus encore en vie in natura est estimé à moins de
250, faisant du Vison d'Europe une espèce très rare. Afin de lutter
contre la disparition de cette espèce, de nombreux programmes ont vu le
jour en France. Ce sont trois Plans Nationaux d'Actions (PNA) sur les
périodes 1999-2003 (DIREN et GREGE, 1999), 2007-2011 (DIREN et GEREA,
2007), et 2021-2031 (DREAL et al., 2021), un Plan National d'Actions dit
« intermédiaire » (PNAi) de 2015 à 2021 (DREAL et ONCFS, 2015), et deux
projets européens de conservation, LIFE VISON de 2017 à 2023 (LPO et
al., 2017) et LIFE KANTAURIBAI de 2022 à 2027 (GAN-NIK et al., 2022) qui
ont été mis en place et dont certains sont encore en cours. Ces projets
ont des envergures géographiques qui peuvent différer, les PNA
s'étendant sur 11 départements de trois régions françaises
(Nouvelle-Aquitaine : Charente, Charente-Maritime, Dordogne, Gironde,
Landes, Lot-et-Garonne, Pyrénées-Atlantiques, Deux-Sèvres ; Occitanie :
Gers, Hautes-Pyrénées ; Pays de la Loire : Vendée), le LIFE VISON
s'étendant sur huit sites Natura 2000 du bassin de la Charente
(départements de Charente et Charente-Maritime) et le LIFE KANTAURIBAI
sur 3 sites Natura 2000 du réseau hydrographique du Golfe de Gascogne.

Malgré la caractère très rare et cryptique de l'espèce, les
connaissances sur l'écologie spatiale et le régime alimentaire du Vison
d'Europe sont en partie documentées, mais certaines lacunes demeurent.
Le Vison d'Europe est un mammifère semi-aquatique qui change de gîte
quasi quotidiennement au sein de son grand domaine vital. En moyenne,
les distances parcourues par le Vison d'Europe entre deux localisations
journalières consécutives sont de l'ordre du kilomètre -- 0,4 kilomètres
pour les femelles et 1,8 kilomètres pour les mâles (Palazón et
Ruiz-Olmo, 1998 ; Fournier et al., 2008 ; Cazaillon, 2021). Le sujet
nécessite cependant d'être approfondi, notamment dans l'état actuel très
dégradé des populations et avec des nouveaux outils d'analyses. En
particulier, des périodes correspondant aux variations de déplacements
n'ont pas été étudiées. Le domaine vital du Vison d'Europe s'étend sur
environ une dizaine de kilomètres de cours d'eau chez les mâles et moins
de la moitié pour les femelles (Garin et al., 2002 ; Ceña, 2003 ;
Fournier et al., 2008 ; Palomares et al., 2017a). Aucune donnée n'a
cependant été publiée sur des individus évoluant en marais littoraux
comme ceux hébergeant les derniers noyaux populationnels français. Les
méthodes de modélisation surfacique du domaine vital les plus couramment
utilisées par les auteurs (méthode des Polygones Convexes Minimums, MCP,
ou méthode des Kernels) ne semblent pas adaptées aux configurations
linéaires des rivières, car elles ne prennent pas en compte la sinuosité
des cours d'eau. De plus, l'occupation fine de l'espace, en particulier
la notion de « zone coeur », est peu renseignée pour cette espèce. La
zone coeur correspondant à une zone fortement utilisée et
statistiquement plus utilisée que les zones fortement utilisées dans
l'hypothèse d'une occupation aléatoire de l'espace (Powell, 2000). En ce
qui concerne son habitat, le Vison d'Europe est strictement inféodé aux
zones humides, étant le plus souvent observé dans des zones proches de
l'eau (Palazón, 1998 ; Fournier et al., 2007). L'espèce est connue pour
gîter majoritairement dans la ripisylve lorsque celle-ci est présente,
avec des gîtes soit souterrains, soit au sol dans la végétation dense
comme les ronciers (Zabala et al., 2003 ; Fournier et al., 2007 ;
Palomares et al., 2017b). Ces connaissances sur l'utilisation de
l'habitat doivent cependant être approfondies. Enfin, son régime
alimentaire est constitué de micromammifères, d'oiseaux, de poissons,
d'amphibiens et d'invertébrés aquatiques dans des proportions qui
peuvent changer entre différents pays (Sidorovich et al., 1998 ; Palazón
et al., 2004 ; Palazón et al., 2008), voire au sein d'une zone définie
(Palazón et al., 2004). Il n'existe cependant aucune étude publiée du
régime alimentaire du Vison d'Europe en France et les études publiées
dans d'autres pays ne prennent pas en compte l'écologie spatiale du
Vison d'Europe pour expliquer les variations de régime alimentaire.

Dans ce contexte, le projet que nous souhaitons mener a pour objectif de
mettre à jour les connaissances sur l'écologie du Vison d'Europe, grâce
à de nouveaux protocoles et/ou de nouvelles méthodes d'analyses sur les
données déjà existantes. L'amélioration des connaissances sur la
mobilité de l'espèce permettra également d'estimer les facteurs
écologiques influençant les risques de collisions routières, facteur
majeur de surmortalité pour les derniers noyaux populationnels. Les
informations apportées par les nouvelles analyses menées permettront
d'orienter plus précisément les stratégies actuelles de conservation du
Vison d'Europe en France, dans son milieu naturel. Les résultats de ce
projet contribueront de surcroît à la définition des meilleures
conditions possibles requises pour de futures translocations d'individus
dans le milieu naturel.

\bookmarksetup{startatroot}

\chapter{Méthodologie}\label{muxe9thodologie}

\section{Occupation de l'espace}\label{occupation-de-lespace}

Les domaines vitaux de tous les individus seront modélisés grâce à des
nouvelles méthodes qui n'ont pas été utilisées jusqu'à présent pour le
Vison d'Europe (Local Convex Hull (LoCoH), ponts browniens\ldots), à
partir des localisations quotidiennes de chaque individu. Les surfaces
occupées ainsi estimées par chacune des modélisations seront comparées
avec celles des modélisations les plus communément utilisées par les
auteurs ayant travaillé sur l'espèce (MCP, Kernels). L'analyse devrait
permettre de proposer la meilleure méthode à retenir pour la
modélisation des domaines vitaux du Vison d'Europe, tout en tenant
compte de la configuration bien différente des zones de marais et des
vallées alluviales sinueuses. Les domaines vitaux seront ensuite
analysés en fonction des caractéristiques des individus (sexe, classe
d'âge, statut reproducteur\ldots) par modèles mixtes en première
intention car en effet ces modèles devront prendre en compte comme
variable aléatoire entre autres la différence de temporalité entre les
deux projets mais aussi d'autres facteurs temporaires (saisons\ldots).
Des tests subsidiaires de choix de modèle (Critère d'Information
d'Akaike\ldots) viendront compléter les analyses. La modélisation des
surfaces exploitées sera également étudiée en fonction de la dispersion
spatiale des localisations, afin de définir des zones plus ou moins
utilisées au sein du domaine vital. Ces approches devraient donc
permettre de mettre en évidence des « zones coeurs », c'est-à-dire des
surfaces statistiquement plus utilisées que les zones fortement
utilisées dans l'hypothèse d'une utilisation aléatoire de l'espace
(Powell, 2000).

\section{Patrons de déplacements}\label{patrons-de-duxe9placements}

Grâce aux suivis continus des individus dans les Landes de Gascogne, une
étude de la trajectométrie pourra être faite afin de décrire des
typologies de déplacements selon les différentes phases d'activité
(chasse, déplacement entre gîte, \ldots). Les modèles utilisés pour
décrire ces phases d'activité devront prendre en compte certaines
caractéristiques des individus (âge, sexe, statut reproducteur, \ldots)
et certaines caractéristiques spatiales relevées lors de la partie i. La
mobilité individuelle sera également étudiée d'une autre manière à
partir des données qui correspondent aux localisations quotidiennes. En
effet, Laundré et al.~(1987) ont montré qu'utiliser des distances entre
localisations relevées à un jour d'intervalle présentaient certains
problèmes en tant qu'indicateur du trajet et des mouvements totaux d'un
individu. De la trajectométrie ne peut donc pas être fait avec ces
données et l'étude de la mobilité individuelle à partir des
localisations quotidiennes sera donc menée en utilisant comme indicateur
les distances entre deux localisations diurnes relevées à un jour
d'intervalle. La modélisation de cet indicateur se fera en projetant
chaque gîte sur l'axe médian du lit majeur du cours d'eau utilisé et en
calculant la distance entre le gîte d'un jour et celui de la veille en
suivant cet axe médian. Cette méthode est proposée pour venir se
substituer aux distances euclidiennes, afin de mieux correspondre à la
réalité du terrain et à la sinuosité des cours d'eau. Ensuite, les
distances seront comparées sur des échelles temporelles différentes
(jour, semaine, mois, saison\ldots). Les distances seront aussi
comparées en fonction du sexe, voire de l'âge et du statut reproducteur.
De la même manière que pour l'étude de l'occupation de l'espace, les
analyses seront des modèles prenant en compte les différentes variables
d'influence, complétées par des analyses de choix de modèles. Ainsi, des
périodes de plus ou moins forte mobilité seront définies et pourront
être reliées à certains évènements (rut, reproduction, élevage des
jeunes\ldots). Afin de pouvoir relier les variations des distances
parcourues et les évènements écologiques cités juste avant, il faudra
prendre en compte les gîtes de reproduction dans cette étude. Ces
périodes pourront être définies comme des périodes écologiques de
l'espèce. En outre, la mobilité au sein des différentes parties des
domaines vitaux sera également étudiée grâce aux résultats de la partie
i.

\section{Utilisation de l'habitat}\label{utilisation-de-lhabitat}

Il s'agira d'identifier les habitats utilisés et ceux sélectionnés par
l'espèce pour installer ses gîtes diurnes grâce à la cartographie de
l'occupation du sol réalisée. Deux approches complémentaires seront
menées : 1) analyser les modalités d'utilisation des habitats en
comparant les distributions des gîtes par habitat grâce à une analyse
multivariée en composantes, 2) définir les sélections d'habitats en
comparant en première intention les habitats des localisations à la
disponibilité présente dans le domaine vital, c'est-à-dire une sélection
du 3ème ordre (sensu Johnson, 1980). Les analyses seront faites grâce à
une approche de type k-select (Calenge et al., 2005). L'utilisation et
la sélection des habitats seront également analysées en fonction des
caractéristiques des individus (sexe, classe d'âge, statut
reproducteur\ldots). De plus, une analyse temporelle pourra être menée
en utilisant entre autres les périodes définies lors de l'analyse de la
mobilité (partie ii.). En outre, les compositions en habitats au sein
des différentes surfaces décrites lors des analyses de l'occupation de
l'espace (partie i.) mais aussi lors de la description des surfaces de
chasse dans l'étude de la mobilité (partie ii.) seront également
étudiées. Une analyse spatiale de l'utilisation et de la sélection des
habitats sera ainsi menée. Dans cette partie, les gîtes de repos et les
gîtes de mises bas et d'élevage des jeunes ne seront pas approchés de la
même façon, le choix du gîte de reproduction impliquant des critères
différents (sécurité des jeunes, \ldots). Une analyse descriptive de
l'habitat utilisé pour installer le gîte de reproduction sera ainsi
menée en parallèle, étant donné le faible nombre de gîtes de
reproduction identifiés.

\section{Caractéristiques des
gîtes}\label{caractuxe9ristiques-des-guxeetes}

Il s'agira d'identifier les caractéristiques préférées dans
l'établissement du gîte diurne. A partir des paramètres environnementaux
relevés à une échelle fine sur chacun des gîtes identifiés par approche
pédestres et cités dans la méthodologie, la préférence de certains de
ces paramètres sera analysée selon les caractéristiques des individus
(sexe, classe d'âge, statut reproducteur\ldots). L'analyse devra
également prendre en compte l'habitat dans lequel se situe le gîte
diurne afin de relever de potentiels correspondances entre les deux
variables. Une analyse multivariée de type canonique est pensée comme
première approche (analyse procustéenne ou analyse de co-inertie).
Enfin, des facteurs temporels (périodes définies lors de l'analyse de la
mobilité partie ii.) et spatiaux (surfaces décrites dans l'étude de
l'occupation de l'espace partie i.) devront être pris en compte pour
expliquer les choix de certaines caractéristiques. De même que dans
l'étude de l'habitat, les gîtes de repos et les gîtes de mises bas et
d'élevage des jeunes ne seront pas approchés de la même façon. Une
analyse descriptive des caractéristiques préférées pour installer le
gîte de reproduction sera ainsi menée en parallèle, pour les mêmes
raisons que pour l'étude de l'habitat.

\section{Régime alimentaire}\label{ruxe9gime-alimentaire}

Pour finir, le régime alimentaire sera étudié grâce aux analyses de la
composition de crottes récoltées sur les gîtes identifiés lors du
radiopistage. Dans un premier temps, des indices de diversité ou de
richesse (basé sur les indices de Shannon et de Simpson) seront calculés
pour chacune des crottes et ces indices seront comparés au sein des
habitats (partie iii.) et des surfaces (partie i.) dans lesquels les
crottes sont trouvées. Le Vison d'Europe pouvant chasser entre deux
gîtes diurnes, une attention particulière devra être porté à l'échelle à
laquelle se feront les analyses. De plus, les études actuelles semblent
montrer que cette espèce est généraliste, il faudra donc mettre en avant
les habitats ou les surfaces dans lesquels les indices montrent une
grande diversité d'espèces voire des taxons supérieurs. Ensuite, et sur
le même principe que pour l'habitat, l'étude du régime alimentaire
tentera de mettre en avant une sélection de certaines proies par les
différents individus selon la ressource disponible dans le domaine
vital. Il s'agira de comparer les disponibilités en proies relevées lors
des inventaires aux proportions de proies retrouvées dans les fèces par
une approche s'inspirant des méthodes analytiques de sélection des
habitats (k-select, eigenanalysis, \ldots). L'individu, le sexe et l'âge
seront des variables à prendre en compte pour expliquer une potentielle
sélection. Cette comparaison devra également prendre en compte la
période (résultats de la partie ii.) puisque les proies du Vison
d'Europe n'ont pas toutes la même écologie. Enfin, les crottes ayant été
localisées, il est possible d'observer des différences de proies
sélectionnées en fonction de la localisation au sein du domaine vital
(dans ou hors zone coeur définies en partie i.).

\section{Risque de collisions
routières}\label{risque-de-collisions-routiuxe8res}

Pour cette analyse, des variables permettant d'estimer un risque de
collision devront tout d'abord être identifiées. Ces variables prendront
en compte les résultats des analyses précédentes mais pas seulement. En
effet le premier type de variable se concentrera sur les composantes
anthropiques (trafic routier sur le franchissement, distance au prochain
franchissement le plus proche\ldots). L'autre type de variables
concernera des variables environnementales définies à partir des
analyses précédentes, en particulier les résultats de l'analyse de
l'utilisation de l'habitat partie iii (présence/absence d'habitats
favorables dans un rayon défini par la mobilité journalière,
présence/absence d'habitats favorables des deux côtés du franchissement
et distance à ceux-ci\ldots). Une analyse précise des collisions
routières recensées lors des différents projets permettra d'associer à
chacune des collisions des valeurs pour chacun des variables retenues.
Une analyse multivariée en composante (analyse des correspondances
multiples ou analyse mixte) permettra ensuite d'expliquer les facteurs
principaux expliquant la mortalité par collision routière.

\bookmarksetup{startatroot}

\chapter{Résultats}\label{ruxe9sultats}

\bookmarksetup{startatroot}

\chapter*{References}\label{references}
\addcontentsline{toc}{chapter}{References}

\markboth{References}{References}

\phantomsection\label{refs}
\begin{CSLReferences}{0}{1}
\end{CSLReferences}



\end{document}
